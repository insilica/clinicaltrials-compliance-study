\documentclass[10pt]{article}

\usepackage{hyperref}

\usepackage{lmodern}
\usepackage{amsmath}
\usepackage{xcolor}
\usepackage{graphicx}
\usepackage{comment}

\usepackage[
	style=nejm,
	backend=biber,
	doi=true,
	maxcitenames=1,mincitenames=1,
]{biblatex}

\usepackage{newunicodechar}
%% Warning: Missing character: There is no … (U+2026) in font rm-lmtt8!
\newunicodechar{^^^^2026}{\ldots} % Mapping character: …
%% Warning: Missing character: There is no – (U+2013) in font rm-lmtt8!
\newunicodechar{^^^^2013}{--} % Mapping character: –
%% Warning: Missing character: There is no — (U+2014) in font rm-lmr10!
\newunicodechar{^^^^2014}{---} % Mapping character: —

\newunicodechar{^^^^00a7}{\S} % Mapping character: §

\usepackage{enumitem}
\newenvironment{abstractdesc}{%
	\begin{description}[align=left, labelindent=0pt, leftmargin=0pt, style=nextline]
	}{%
	\end{description}
	}

\usepackage{listings}
\usepackage{listingsutf8}

%% Define language for YAML
\lstdefinelanguage{YAML}{
	keywords={true,false,null,y,n},
	ndkeywords={},
	sensitive=false,
	comment=[l]{\#},
	morestring=[b]',
	morestring=[b]"
}

\input{report/inc/listings/color-scheme.tex}

\lstset{
	basicstyle=\ttfamily\footnotesize,
	%%
	%% Character encoding handling
	%inputencoding=utf8, % Not needed with UTF-8 TeX engine
	extendedchars=true,
	literate      =        % Support additional characters
		{§}{{\S}}1
		{–}{{--}}1
		{—}{{---}}1
		{…}{{$\ldots$}}1
		{→}{{\rightarrow}}1
		,
	%%
	%% Spacing
	tabsize=2,
	breakatwhitespace=false,
	breaklines=true,
	postbreak=\mbox{\textcolor{red}{$\hookrightarrow$}\space},
	keepspaces=true,
	%%
	%% Numbering
	numbers=left,
	numbersep=5pt,
	%%
	%% Space formatting
	showspaces=false,
	showstringspaces=false
	showtabs=false,
	%%
	%% Decorations
	columns=fullflexible,
	frame=single,
}

\lstdefinestyle{RvPerl}{language=Perl,
	rangeprefix=\#\#\ \{\{\ ,
	rangesuffix=\ \}\},
	includerangemarker=false,
}

\lstdefinestyle{RvSQL}{language=SQL,
	rangeprefix=--\#\#\ \{\{\ ,
	rangesuffix=\ \}\},
	includerangemarker=false,
}

\lstdefinestyle{RvR}{language=R,
	rangeprefix=\#\#\ \{\{\ ,
	rangesuffix=\ \}\},
	includerangemarker=false,
}

\lstdefinestyle{RvYAML}{language=YAML,
	rangeprefix=\#\#\ \{\{\ ,
	rangesuffix=\ \}\},
	includerangemarker=false,
}

%% For the article class
\usepackage{etoolbox}
%% Redefine the abstract environment to use normal margins
\makeatletter
\patchcmd{\abstract}{\quotation}{}{}{}
\patchcmd{\endabstract}{\endquotation}{}{}{}
\makeatother

\input{report/inc/abbrev.tex}


%% TODO
\title{ClinicalTrials.gov Compliance}
\author{A.U. Thor}
\date{TODO}



\addbibresource{report/biblio.bib}

\begin{document}

\maketitle

\begin{abstract}
\noindent
\begin{abstractdesc}
\item[Background] TODO
%
\item[Methods] TODO
%
\item[Results] TODO
%
\item[Conclusions] TODO
\end{abstractdesc}
\end{abstract}

\tableofcontents


\section{Background}

The establishment of the clinical trial submission requirements in
Section 801 of the Food and Drug Administration Amendments Act of 2007
(FDAAA 801)~\cite{publiclaw_fdaa_2007_s801}
mandated that any applicable clinical trials (ACTs) must
submit their summary results to the \ctgov{} data bank one year from study completion.
This regulation went into effect on \ISOtoLocaleDisplayDate{2017-01-18}
and compliance was required starting 90 days later on
\ISOtoLocaleDisplayDate{2017-04-18} for all responsible
parties~\cite{ctgov_fdaaa801_final_rule_2024,zarin_trial_2016}.

\section{Methods}

\textcite{anderson_compliance_2015} present an algorithm to
determine which trials are likely to be ACTs which they term as
highly likely applicable clinical trials (HLACTs).

To ensure that similar methods are used, the replicated HLACTs
algorithm is compared against the \textcite{anderson_compliance_2015}
shared dataset~\cite{anderson_data_20130927}.

\section{Results}


\begin{figure}
	\includegraphics[width=0.95\textwidth]{figtab/rule-effective/fig.result_reported_within.facet_overall.stacked_area.png}
	\label{fig:results-reported-rule-effective-overall}
	\caption{Results before and after the effective rule date (overall)}
\end{figure}

\begin{figure}
	\includegraphics[width=0.95\textwidth]{figtab/rule-effective/fig.result_reported_within.facet_funding.stacked_area.png}
	\label{fig:results-reported-rule-effective-funding}
	\caption{Results before and after the effective rule date by funding source}
\end{figure}


\section{Discussion}

\printbibliography

\end{document}
